\documentclass[12pt]{article}
\usepackage{charter} % font
\usepackage[margin=1in]{geometry} % margin
\usepackage{hyperref} % hyperlinks
\usepackage{enumitem} % Enumeration
\usepackage{xcolor, soul} % colors
\usepackage{graphicx} % images
\usepackage{amsmath,amsthm,amssymb} % math
\usepackage{float} % image placement

\hypersetup{
	colorlinks,
	linkcolor={red!50!black},
	citecolor={blue!50!black},
	urlcolor={blue!80!black}
}

\theoremstyle{definition}
\newtheorem{definition}{Definition}
\newtheorem{practiceproblem}{Problem}[section]

\def\scratchwork{\vspace*{15em}} % space for problem solving

% Homework Answer links
\def\psetone{https://github.com/zacktraczyk/CSE102-Midterm-Study-Guide/blob/main/hw\%20answers/vaggos\_W\_24\_CSE102\_01\_PSET\_1\_solutions.pdf}
\def\psettwo{https://github.com/zacktraczyk/CSE102-Midterm-Study-Guide/blob/main/hw\%20answers/vaggos\_W\_24\_CSE102\_01\_PSET\_2\_solutions.pdf}
\def\psetthree{https://github.com/zacktraczyk/CSE102-Midterm-Study-Guide/blob/main/hw\%20answers/vaggos\_W\_24\_CSE102\_01\_PSET\_3\_solutions.pdf}


\newenvironment{problem}[2]
{
	\def\linktext{#1}
	\def\linkdest{#2}
	\noindent \begin{minipage}{\textwidth}
		\begin{practiceproblem}
}
{	
		\end{practiceproblem}
		\href{\linkdest}{\textbf{Answer:} \linktext}
		\scratchwork
	\end{minipage}
}

\def\hlgreen{\sethlcolor{green}\hl}
\def\hlyellow{\sethlcolor{yellow}\hl}


%%%%%%%%%%%%%%%%%%%%%%%%%%%%%%%%%%%%%%%%%%%%%%%%%%%%%%%%%%%%%%%%%%%%%%%%%%%%%%%%

\title{CSE 102 - Midterm Study Guide}

\author{Zack Traczyk}
\date{Vaggos, Winter 2024}

%%%%%%%%%%%%%%%%%%%%%%%%%%%%%%%%%%%%%%%%%%%%%%%%%%%%%%%%%%%%%%%%%%%%%%%%%%%%%%%%

\begin{document}

	\maketitle

	\tableofcontents

	\pagebreak


%%%%%%%%%%%%%%%%%%%%%%%%%%%%%%%%%%%%%%%%%%%%%%%%%%%%%%%%%%%%%%%%%%%%%%%%%%%%%%%%

	\section{Introductory Material Review}

	\subsection{Asymptotic Bounds}

	\begin{definition}[Big-O]
		$f(n) = O(g(n))$ if there exists a positive constant $c$ and an integer
		$n_0$ such that $f(n) \leq c \cdot g(n)$ for all $n \geq n_0$.
	\end{definition}

	\begin{definition}[Big-$\Omega$]
		$f(n) = \Omega(g(n))$ if there exists a positive constant $c$ and an
		integer $n_0$ such that $c \cdot g(n) \leq f(n)$ for all $n \geq n_0$.
	\end{definition}

	\begin{definition}[Big-$\Theta$]
		$f(n) = \Theta(g(n))$ if there exists positive constants $c_1$, $c_2$,
		and an integer $n_0$ such that $c_1 \cdot g(n) \leq f(n) \leq c_2 \cdot
		g(n)$ for all $n \geq n_0$.
	\end{definition}

	\subsection{Inductive Proofs}

	\begin{definition}[Inductive Proof]
		Simples rules of induction taken from CSE 16 with Prof. Tracy Larrabee.
		\begin{enumerate}
			\item Write down the \hlyellow{Left Hand Side} of $P(k+1)$.
			\item Rewrite $P(k+1)$ to include \hlyellow{Left Hand Side} $P(k)$. 
			\item Replace \hlyellow{Left Hand Side} of $P(k)$ with
				\hlgreen{Right Hand Side} of $P(k)$.
			\item Rewrite so \hlgreen{Right Hand Side} of $P(k)$ becomes
				\hlgreen{Right Hand Side} of $P(k+1)$.
		\end{enumerate}
	\end{definition}

	\noindent \textbf{Examples:}

	\begin{proof} For all $n \in \mathbb{Z}^+$ the number $n^2 + n$ is even \\
		\textbf{Base Case (n = 1):} 
		$$n^2 + n = 1 + 1 = 2\text{, which is even}$$ \\
		\textbf{Inductive Hypothesis:}
		$$\text{Assume }n^2 + n\text{ is even, prove }(n + 1)^2 + (n + 1)\text{
		is even}$$
		\begin{gather}
			(n + 1)^2 + (n + 1) = n^2 + 2n + 1 + n + 1 \\
			= n^2 + n + 2n + 2 \\
			2p = n^2 + n\text{, 2p is the definition of even} \\
			= 2p + 2n + 2 \\
			= 2(p + n + 1)\text{, which is even}
		\end{gather}
	\end{proof}


	\subsection{Practice Problems}

	\begin{problem}{HW1 - Ex.1 Part 2}{\psetone}
		Prove that $T(n) = 2T(n - 1) + 1$ is $T(n) = 2^n - 1$.
	\end{problem}

%%%%%%%%%%%%%%%%%%%%%%%%%%%%%%%%%%%%%%%%%%%%%%%%%%%%%%%%%%%%%%%%%%%%%%%%%%%%%%%%

	\section{Solving Recurrence Relations}

	\subsection{Master Theorem}

	\subsection{Unpacking Tree / Algebraic Pattern}

	\subsection{Substitution}

	\subsection{Guess and Verify}

	\subsection{Practice Problems}

	Like in many previous exercises and homeworks, find tight asymptotic bounds
	(big-Theta) for $T(n)$ in each of the cases.

	\begin{problem}{HW3 - Ex.4}{\psetthree}
		$T(n)=2T(n/4)+n^2\sqrt n$
	\end{problem}

	\begin{problem}{HW3 - Ex.4}{\psetthree}
		$T(n)=T(n-1)+\frac1n$
	\end{problem}

	\begin{problem}{HW3 - Ex.4}{\psetthree}
		$T(n)=1600T(n/4)+n!$ (hint: answering this shouldn't require too many,
		if any, difficult calculations)
	\end{problem}

	\begin{problem}{HW3 - Ex.4}{\psetthree}
		$T(n)=6T(n/3)+n^4/\log^{25} n$ (hint: answering this shouldn't require
		too many, if any, difficult calculations)
	\end{problem}

	\begin{problem}{HW3 - Ex.4}{\psetthree}
		$T(n)=\sqrt n T(\sqrt{n}) + n$ (hint: when everything fails, you guess
		and check)
	\end{problem}

	\begin{problem}{HW3 - Ex.4}{\psetthree}
		$T(n)=T(n/2)+n(5-\cos^2n\sin^{20}n)$ (hint: answering this shouldn't
		require too many, if any, difficult calculations, just think the most
		basic trigonometric inequality)
	\end{problem}

	\begin{problem}{HW3 - Ex.4}{\psetthree}
		$T(n)=\alpha T(n/4)+n^2$ (hint: your answer should depend on the
		$\alpha$ parameter)
	\end{problem}

	\begin{problem}{HW3 - Ex.4}{\psetthree}
		$T(n)=5T(n/5) +\frac{n}{\log_5 n}$ (hint: think of $n=5^m$. Also the
		recursion $T(n)=T(n-1)+\frac1n$ above may come in handy.)
	\end{problem}

%%%%%%%%%%%%%%%%%%%%%%%%%%%%%%%%%%%%%%%%%%%%%%%%%%%%%%%%%%%%%%%%%%%%%%%%%%%%%%%%

	\section{Algorithms}

	\subsection{Binary Search}

	\subsection{Sorting}

	\subsubsection{Lower Bounds}

	\subsection{Merge Sort}

	\subsection{Number of leaves / depth as proof for lower asymptotic bounds}

	\subsection{Quick Select}

	\subsection{Dynamic Programming}

	\subsubsection{Fibonacci}

	\subsubsection{Binomial Coefficients}

	\subsubsection{Maximize independent set}

\end{document}

